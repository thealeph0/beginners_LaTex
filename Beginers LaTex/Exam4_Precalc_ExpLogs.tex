\documentclass[addpoints,12pt]{exam}
\usepackage{wrapfig} 
\usepackage{graphicx}
\usepackage{amsmath}
\usepackage{amssymb}
\usepackage{wrapfig} %able to insert pictures in text
\usepackage{gensymb} %allows the degree symbol for weather or angles
\usepackage{enumitem}

\usepackage{mathtools,etoolbox}
\DeclarePairedDelimiterX{\abs}[1]{\lvert}{\rvert}{\ifblank{#1}{{}\cdot{}}{#1}}
\newcommand\norm[1]{\left\lVert#1\right\rVert}

\begin{document}

\begin{center}
\fbox{\fbox{\parbox{5.5in}{\centering
Honors Pre-Calculus Exam 4}}}

\end{center}

\noindent I acknowledge that I have read and agree to the Crespi Carmelite High School Academic Integrity Contract (page 24 of the Parent/Student Handbook), and I agree to complete this exam in accordance with the contract's stipulations. 

\vspace{0.5in}
\makebox[\textwidth]{Name and period:\enspace\hrulefill}

\vspace{.5 in}
\makebox[\textwidth]{Signature: \enspace\hrulefill}
\vspace{1in}

\noindent \textbf{DO NOT BEGIN UNTIL INSTRUCTED TO DO SO}

\noindent You have until the end of the period to finish this exam.  If you happen to finish early please have a seat until the end of the period. The exam is worth \textbf{70 points}. You are allowed to write on this exam.

\begin{center}
\combinedgradetable[h][questions]

\end{center}

\newpage


\begin{questions}

\question[10] %1
Given $$f(x) = \cfrac{x+4}{x-4} \mbox{ and } g(x)=4^x$$.

\begin{parts}

\part 
Find $(f \circ g)(x)$ and state the domain.
\vspace{3in}
\part
Find $(g \circ f)(-3)$
\vspace{2in}
\end{parts}

\newpage

\question[10] %2 Evaluate
\textbf{Evaluate:}

	\begin{parts}
		\part $\log_{\tfrac{3}{2}} \bigg(\cfrac{16}{81}\bigg) = $
		\vspace{4cm}
		\part $\log_{\tfrac{1}{3}}(3^{4x}) = $
		\vspace{4cm}
		\part $\log_{6} (-36) = $
		\vspace{4cm}
		\part $\log (1,000,000,000) = $
	\end{parts}
\newpage

\question[10] %3
\textbf{Write in logarithmic form:}

\begin{parts}
	\part $10^5=100,000$
	\vspace{1.3in}
	\part $5^{-2}=\cfrac{1}{25}$
	\vspace{1.3in}
	\part $\bigg(\cfrac{1}{4}\bigg)^{-3}=64$
	\vspace{1.3in}
	\part $y^{2z}=A$
\end{parts}

\newpage

\question[10] %4
Completely expand each logarithm. All exponents should be written as factors. Simplify.

\begin{parts}
	\part $\log_3 (3x\sqrt[3]{x-2})^4$
	\vspace{3in}
	\part $\ln \cfrac{e^x}{x^3(x^2-7)}$
	\vspace{3in}
	\part $\ln \sqrt[5]{\cfrac{a^3b}{c}}$
\end{parts}

\newpage

\question[10] %5
Condense each expression to a single logarithm.
	\begin{parts}
		\part $2\log_3 (6) - \cfrac{3}{2}\log_3 (4) + log_3 (18)$
		\vspace{2in}
		\part $\log_2(5x^2y^3)-\log_2(20x^4y)+\log_2(2xy^6)$
		\vspace{2.5in}
		\part $log \bigg(\cfrac{x^2-9}{x^2+2x}\bigg) - log \bigg(\cfrac{x^2+2x-3}{x^2-x-6}\bigg) $ (Hint: Use log rules and factor each to simplify.)
	\end{parts}

\newpage
	
\question[10]
Solve each of the following equations.

(I will pick 2 of these 3 type of problems)

\begin{parts}
	\part $2^{x-1}=3^{x+1}$
	
	\vspace{5cm}
	
	\part $\log_2 (x-4) + \log_2(x+4) = 3$
	
	\vspace{5cm}
	
	\part $9^x + 4 \cdot 3^x - 3 = 0$ (Equation of Quadratic Type)
\end{parts}

	
\newpage
\question[10]%6
For the graph $$f(x)= \cfrac{1}{4^x}, \mbox{ where }  x \in \mathbb{R}$$
	\begin{parts}
		\part Find the domain and range. (\textbf{Interval or Set-Builder Notation})
		\vspace{3cm}
		\part Graph the function.  \textbf{Make sure to label at least three points.  Make sure any asymptotes are clearly marked and labeled appropriately.}
		\vspace{2.5in}
		\part What is the inverse of this function?
		\vspace{3cm}
		\part What is the domain and range of the inverse function?
	\end{parts}
	
\newpage
\question[10]%7
For the graph $y= \cfrac{1}{2}\log_3(x+4)+2$.
	\begin{parts}
		\part Find the domain and range. (\textbf{Interval or Set-Builder Notation})
		\vspace{3cm}
		\part Describe the transformations necessary to graph the function.
		\vspace{2cm}
		\part Graph the function.  \textbf{Make sure to label at least three points.  Make sure any asymptotes are clearly marked and labeled appropriately.}
		\vspace{2.5in}
		\part What is the inverse of this function?
		\vspace{3cm}
		\part What is the domain and range of the inverse function?
	\end{parts}

		



\end{questions}

\end{document}