\documentclass[12pt, letterpaper]{report}
\usepackage[export]{adjustbox}
\usepackage[margin=0.5in]{geometry} %defined paper below
\usepackage[utf8]{inputenc}
\usepackage{blindtext}
%\usepackage[usenames,dvipsnames]{xcolor} %adds mucho colors

%%%%%%%%%%%%%%%%%%%%%%%%%%%URLs%%%%%%%%%%%%%%%%%%%%%%%%%%%%%%%%%%%%
\usepackage{hyperref}
\hypersetup{
    colorlinks=true,
    linkcolor=blue,
    filecolor=magenta,      
    urlcolor=blue,
    pdfpagemode=FullScreen,
    }

\urlstyle{same} %%keeps the URL font same as text

%%%%%%%%%%%%%%%%%%%%%%%%%%%%%%%%%%%%%%%%%%%%%%%%%%%%%%%%%%%%%%%%%%%%
%%%%%%%%%%%%%%%%%%%%%%%%%%%Math Package%%%%%%%%%%%%%%%%%%%%%%%%%%%%%

\usepackage{amsmath}
\usepackage{amssymb}
\usepackage{amsthm}
\usepackage{graphicx}
\usepackage[mathscr]{eucal} %produce a script H, (for Hilbert Space or Laplace Transform)
\usepackage{tasks} %allows nice lists https://tex.stackexchange.com/a/324740

%%%%%%%%%%%%%%%%%%%%%%%%%%%%%%%%%%%%%%%%%%%%%%%%%%%%%%%%%%%%%%%%%%%%
%%%%%%%%%%%%%%%%%%%%%%%%%%%%%%%%%%%%%%%%%%%%%%%%%%%%%%%%%%%%%%%%%%%%

%%%%%%%%%included for CS%%%%%%%%%%%%%%%%%%%%%%

\usepackage{listings}
\usepackage{color}

\definecolor{dkgreen}{rgb}{0,0.6,0}
\definecolor{gray}{rgb}{0.5,0.5,0.5}
\definecolor{mauve}{rgb}{0.58,0,0.82}

\lstset{frame=tb,
  language={[latex]TeX}, %%change to your preferred language
  aboveskip=1mm,
  belowskip=1mm,
  showstringspaces=false,
  columns=flexible,
  basicstyle={\small\ttfamily},
  numbers=left,
  numberstyle=\tiny\color{gray},
  keywordstyle=\color{blue},
  commentstyle=\color{dkgreen},
  stringstyle=\color{mauve},
  breaklines=true,
  breakatwhitespace=true,
  tabsize=3
}

%%%%%%%%%%%%end of CS additions%%%%%%%%%%%%%%%



%%%%%%%%%%%%%%%%%%%%%%%%%%%%%%%%%%%%%%%%%%%%%%%%%%%%%%%%%%%%%%%%%%%%
%%%%%%%%%%%%%%%%%%%%%%%%%%%%%%%%%%%%%%%%%%%%%%%%%%%%%%%%%%%%%%%%%%%%

\usepackage{enumitem} %allows to resume enumeration
\usepackage{tabu} %allows tables using tabu
\usepackage{standalone}
\usepackage{tasks}

\usepackage{tikz}
\usetikzlibrary{decorations.pathreplacing}
\usepackage{tikz-network}
\usetikzlibrary{matrix}

\usetikzlibrary{arrows.meta,shapes,automata,backgrounds,petri,positioning}
\usetikzlibrary{decorations.pathmorphing}
\usetikzlibrary{decorations.shapes}
\usetikzlibrary{decorations.text}
\usetikzlibrary{decorations.fractals}
\usetikzlibrary{decorations.footprints}
\usetikzlibrary{calc}
\usetikzlibrary{spy}
\usetikzlibrary{arrows}
\usepackage{mathtools,etoolbox}

\usepackage{parskip} %something about paragraph lines
%defines columns
\usepackage{multicol}
\setlength{\columnsep}{1cm}


\usepackage{wrapfig} %able to insert pictures in text
\usepackage{gensymb} %allows the degree symbol for weather or angles
\usepackage{epigraph} %adds little quote start of chapter
\usepackage{systeme}%something about allowing system of equations stuff

\usepackage{pdfpages} %allows for import of separate files


\usepackage[framemethod=TikZ]{mdframed} %allows framed definitions

%%%%%%%%%%%%%%%%%%%%%%THEOREM BOXES%%%%%%%%%%%%%%%%%%%%%%%%%%%%%%
\newcounter{theo}\setcounter{theo}{0}
%\renewcommand{\thetheo}{\arabic{chapter}.\arabic{theo}}
\newenvironment{theo}[2][]{%
						\refstepcounter{theo}%
						\ifstrempty{#1}%
						{\mdfsetup{%
						frametitle={%
						\tikz[baseline=(current bounding box.east),outer sep=0pt]
						\node[anchor=east,rectangle,fill=JungleGreen!55]
						{\strut Theorem~\thetheo};}}
						}%
					{\mdfsetup{%
					frametitle={%
					\tikz[baseline=(current bounding box.east),outer sep=0pt]
					\node[anchor=east,rectangle,fill=JungleGreen!55]
					{\strut Theorem~\thetheo:~#1};}}%
						}%
					\mdfsetup{innertopmargin=10pt,linecolor=JungleGreen!55,%
					linewidth=2pt,topline=true,%
					frametitleaboveskip=\dimexpr-\ht\strutbox\relax
					}
\begin{mdframed}[]\relax%
\label{#2}}{\end{mdframed}}
%%%%%%%%%%%%%%%%%%%%%%%%%%%%%%%%%%%%%%%%%%%%%%%%%%%%%%%%%%%%%%%%

%%%%%%%%%%%%%%%%%%%%%%DEFINTION BOXES%%%%%%%%%%%%%%%%%%%%%%%%%%%
\newcounter{defi}[chapter] \setcounter{defi}{0}
\renewcommand{\thedefi}{\arabic{chapter}.\arabic{defi}}
\newenvironment{defi}[2][]{%
						\refstepcounter{defi}%
						\ifstrempty{#1}%
						{\mdfsetup{%
						frametitle={%
						\tikz[baseline=(current bounding box.east),outer sep=0pt]
						\node[anchor=east,rectangle,fill=WildStrawberry!55]
						{\strut Definition~\thedefi};}}
						}%
					{\mdfsetup{%
					frametitle={%
					\tikz[baseline=(current bounding box.east),outer sep=0pt]
					\node[anchor=east,rectangle,fill=WildStrawberry!55]
					{\strut Definition~\thedefi:~#1};}}%
						}%
					\mdfsetup{innertopmargin=10pt,linecolor=WildStrawberry!55,%
					linewidth=2pt,topline=true,%
					frametitleaboveskip=\dimexpr-\ht\strutbox\relax
					}
\begin{mdframed}[]\relax%
\label{#2}}{\end{mdframed}}
%%%%%%%%%%%%%%%%%%%%%%%%%%%%%%%%%%%%%%%%%%%%%%%%%%%%%%%%%%

%%%%%%%%%%%%%%%% Steps %%%%%%%%%%%%%%%%%%%%%%%%%%%%%%%%%%%

\newcounter{proc}[section] \setcounter{proc}{0}
\renewcommand{\theproc}{ }
\newenvironment{proc}[2][]{%
						\refstepcounter{proc}%
						\ifstrempty{#1}%
						{\mdfsetup{%
						frametitle={%
						\tikz[baseline=(current bounding box.east),outer sep=0pt]
						\node[anchor=east,rectangle, rounded corners,fill=RoyalBlue!60]
						{\strut PS~};}}
						}%
					{\mdfsetup{%
					frametitle={%
					\tikz[baseline=(current bounding box.east),outer sep=0pt]
					\node[anchor=east,rectangle, rounded corners,fill=RoyalBlue!60]
					{\strut PS~:~#1};}}%
						}%
					\mdfsetup{roundcorner=5pt, innertopmargin=10pt,linecolor=RoyalBlue!60, %
					linewidth=2pt,topline=true,%
					frametitleaboveskip=\dimexpr-\ht\strutbox\relax
					}
\begin{mdframed}[]\relax%
\label{#2}}{\end{mdframed}}
%%%%%%%%%%%%%%%%%%%%%%%%%%%%%%%%%%
%%%%%%%%%% Custom Section %%%%%%%%
\usepackage{titlesec}
  \titleformat{\section}
    {\normalfont\bfseries}{\thesection}{1em}{}[{\titlerule[.8pt]}]


%%%%%%%%%%%%%%%%%%%%%%%%%%%%% Tasks%%%%%%%%%%%%%%%%%%%%%%%%%%%%%%%%
\settasks{label-format={\color{green!70!black}\bfseries}, label-align=center, label-offset={5mm}, label-width={10mm}, item-indent={2mm}, item-format={\scshape\small}, column-sep={3mm}, after-item-skip=1mm, after-skip={3mm}
}
%%%%%%%%%%%%%%%%%%%%%%%%%%%%%%%%%%%%%%%%%%%%%%%%%%%%%%%%%%%%%%%%%%%


\newcommand{\ds}{\displaystyle}
\newcommand{\ra}{\rightarrow}
\newcommand{\Ra}{\Rightarrow}
\newcommand{\Lra}{\Leftrightarrow}
\newcommand{\mc}{\mathcal}
\newcommand{\mrm}{\mathrm}
\newcommand{\ul}{\underline}
\newcommand{\ol}{\overline}
\newcommand{\cl}{\overline}
\newcommand{\hs}{\hspace}
\newcommand{\vs}{\vspace}

%%%%%%%%%%%%%%%%%%%Topology%%%%%%%%%%%%%%%%%%%%%%%%%%%%
\newcommand{\interior}[1]{\mathrm{int(#1)}}
\newcommand{\bdy}[1]{\mathrm{bdy(#1)}}
\newcommand{\inter}[1]{#1^{\mathsf{o}}}
%\newcommand{\closure}[1]{\ol{#1}} % simple
\newcommand{\closure}[2][3]{{}\mkern#1mu\ol{\mkern-#1mu#2}} %nicer on eyes
\DeclarePairedDelimiterX{\abs}[1]{\lvert}{\rvert}{\ifblank{#1}{{}\cdot{}}{#1}} %absolute value bars

\def\T{\mathcal{T}}
\def\K{\mathcal{K}}
\def\s{\mathcal{S}}
\def\U{\mathscr{U}}
\def\V{\mathscr{V}}
\def\B{\mathcal{B}}

\newcommand{\metspace}[2]{({#1},{#2})-\mbox{metric space}}
\newcommand{\ms}[2]{({#1},{#2})}
\newcommand{\ts}[2]{({#1},{#2})}

\newcommand{\genfunc}[3]{{#3}\colon {#1} \longrightarrow {#2}}
\newcommand{\genfuncdef}[5]{{#3} \colon {#1} \longrightarrow {#2}\mbox{ where } {#4} \mapsto {#5} }


%%%%%%%%%%%%%%%%%% General %%%%%%%%%%%%%%%%%%%%%%%%%%%%


\newcommand{\rsphere}{\mathbb{C}_\infty}
\newcommand{\Disk}{\mathbb{D}}
\newcommand{\disk}{\Delta}
\newcommand{\wh}{\widehat}
\newcommand{\e}{\varepsilon}
\newcommand{\0}{\emptyset}

\newcommand{\bd}{\partial}
\newcommand{\im}{\mathrm{Im}}
\newcommand{\pr}{\mathrm{Pr}}
\newcommand{\sh}{\mathrm{Sh}}
\newcommand{\dm}{\mathrm{diam}}
\newcommand{\dist}{\mathrm{d}}
\newcommand{\sm}{\setminus}

\newcommand{\tp}[1]{\textcolor{violet}{\textbf{\emph{#1}}}}
\newcommand{\tb}[1]{\textcolor{TealBlue!80!black}{\textbf{\emph{#1}}}}


\def\C{\mathbb{C}}
\def\R{\mathbb{R}}
\def\Q{\mathbb{Q}}
\def\N{\mathbb{N}}
\def\Z{\mathbb{Z}}




%defined theorems and definitions section
\theoremstyle{theorem}
\newtheorem{defn}{\emph{Definition}}

\theoremstyle{definition}
\newtheorem{ndefn}{Naive Definition}

\theoremstyle{definition}
\newtheorem{ex}{Example}

\theoremstyle{definition}
\newtheorem{DIY}{DIY}

\theoremstyle{definition}
\newtheorem{post}{Postulate}

\theoremstyle{theorem}
\newtheorem{thm}{Theorema}

\theoremstyle{theorem}
\newtheorem{cor}{Corollary}

\theoremstyle{remark}
\newtheorem{rmk}{Remark}

\begin{document}

\begin{center}
\Large Linear Fraction Integrals \\
\end{center}

Compute the following anti-derivatives.  See if you identify any patterns.
\begin{multicols}{3}
\bfseries{\begin{Large}
	\begin{enumerate}
		\item $\displaystyle\int \cfrac{1}{x}\; dx$
		\vspace{3.5cm}
		\item $\displaystyle\int \cfrac{1}{y}\; dy$
		\vspace{3.5cm}
		\item $\displaystyle\int \cfrac{1}{x-1}\; dx$
		\vspace{3.5cm}
		\item $\displaystyle\int \cfrac{1}{2t-1}\; dt$
		\vspace{3.5cm}
		\item $\displaystyle\int \cfrac{1}{5t}\; dt$
		\vspace{3.5cm}
		\item $\displaystyle\int \cfrac{1}{s-15}\; ds$
		\vspace{3.5cm}
		\item $\displaystyle\int \cfrac{17}{5t-1}\; dt$
		\vspace{3.5cm}
		\item $\displaystyle\int \cfrac{20}{y-5}\; dy$
		\vspace{3.5cm}
		\item $\displaystyle\int \cfrac{1}{x^2}\; dx$
		\vspace{3.5cm}
		\item $\displaystyle\int \cfrac{1}{1+y^2}\; dy$
		\vspace{3.5cm}
		\item $\displaystyle\int \cfrac{1}{x^4}\; dx$
		\vspace{3.5cm}
		\item $\displaystyle\int \cfrac{1}{x^4+1}\; dx$
	\end{enumerate}
\end{Large}}

\end{multicols}

\includegraphics[width=.3\textwidth,right]{images/intno}

\newpage
\begin{center}
\Large Linear Fraction Integrals \\ (Linear Denominator = $\ln \abs{f}$ + C)
\end{center}

Compute the following anti-derivatives.  See if you identify any patterns.
\begin{multicols}{3}
\bfseries{\begin{Large}
	\begin{enumerate}
		\item $\displaystyle\int \cfrac{1}{x}\; dx$
		\vspace{3cm}
		\item $\displaystyle\int \cfrac{1}{y}\; dy$
		\vspace{3cm}
		\item $\displaystyle\int \cfrac{1}{x+1}\; dx$
		\vspace{3cm}
		\item $\displaystyle\int \cfrac{1}{2t+1}\; dt$
		\vspace{3cm}
		\item $\displaystyle\int \cfrac{1}{5t}\; dt$
		\vspace{3cm}
		\item $\displaystyle\int \cfrac{1}{3s+15}\; ds$
		\vspace{3cm}
		\item $\displaystyle\int \cfrac{17}{5t+4}\; dt$
		\vspace{3cm}
		\item $\displaystyle\int \cfrac{20}{y+5}\; dy$
		\vspace{3cm}
		\item $\displaystyle\int \cfrac{1}{x^2}\; dx$
		\vspace{3cm}
		\item $\displaystyle\int \cfrac{1}{1+y^2}\; dy$
		\vspace{3cm}
		\item $\displaystyle\int \cfrac{1}{x^2-1}\; dx$
		\vspace{3cm}
		\item $\displaystyle\int \cfrac{1}{x^3}\; dx$
	\end{enumerate}
\end{Large}}

\end{multicols}

\newpage

\begin{center}
\Large Exponential Integrals \\
\end{center}

Compute the following anti-derivatives.  See if you identify any patterns.
\begin{multicols}{3}
\bfseries{\begin{Large}
	\begin{enumerate}
		\item $\displaystyle\int e^{x}\; dx$
		\vspace{3.5cm}
		\item $\displaystyle\int e^{2x}\; dx$
		\vspace{3.5cm}
		\item $\displaystyle\int e^{5x}\; dx$
		\vspace{3.5cm}
		\item $\displaystyle\int e^{3x}\; dx$
		\vspace{3.5cm}
		\item $\displaystyle\int e^{5x-1}\; dx$
		\vspace{3.5cm}
		\item $\displaystyle\int e^{3x-20}\; dx$
		\vspace{3.5cm}
		\item $\displaystyle\int e^{2x-7}\; dx$
		\vspace{3.5cm}
		\item $\displaystyle\int e^{\frac{1}{4}x}\; dx$
		\vspace{3.5cm}
		\item $\displaystyle\int e^{-3x}\; dx$
		\vspace{3.5cm}
		\item $\displaystyle\int -3e^{4x}\; dx$
		\vspace{3.5cm}
		\item $\displaystyle\int -2e^{-4x}\; dx$
		\vspace{3.5cm}
		\item $\displaystyle\int 6e^{-6x}\; dx$
		\vspace{3.5cm}
		
	\end{enumerate}
\end{Large}}

\end{multicols}


%
%\newpage
%\begin{center}
%\Large Notes
%\end{center}

\newpage
\begin{center}
\Large Integration of $\sin^{-1}(x)$\\ Homework
\end{center}

Compute the following anti-derivatives.  See if you identify any patterns.
\begin{multicols}{3}
\bfseries{\begin{Large}
	\begin{enumerate}
		\item $\displaystyle\int \cfrac{1}{\sqrt{1-x^2}} \; dx$
		\vspace{3cm}
		\item $\displaystyle\int \cfrac{1}{\sqrt{1-4x^2}} \; dx$
		\vspace{3cm}
		\item $\displaystyle\int \cfrac{1}{\sqrt{1-3x^2}} \; dx$
		\vspace{3cm}
		\item $\displaystyle\int \cfrac{1}{\sqrt{1-16x^2}} \; dx$
		\vspace{3cm}
		\item $\displaystyle\int \cfrac{1}{\sqrt{4-x^2}} \; dx$
		\vspace{3cm}
		\item $\displaystyle\int \cfrac{1}{\sqrt{3-x^2}} \; dx$
		\vspace{3cm}
		\item $\displaystyle\int \cfrac{1}{\sqrt{2-x^2}} \; dx$
		\vspace{3cm}
		\item $\displaystyle\int \cfrac{1}{\sqrt{16-x^2}} \; dx$
		\vspace{3cm}
		\item $\displaystyle\int \cfrac{1}{\sqrt{25-9x^2}} \; dx$
		\vspace{3cm}
		\item $\displaystyle\int \cfrac{1}{\sqrt{4-81x^2}} \; dx$
		\vspace{3cm}
		\item $\displaystyle\int \cfrac{1}{\sqrt{9-16x^2}} \; dx$
		\vspace{3cm}
		\item $\displaystyle\int \cfrac{1}{\sqrt{1-x^2}} \; dx$
	\end{enumerate}
\end{Large}}

\end{multicols}


\newpage
\begin{center}
\Large Integration of $\tan^{-1}(x)$\\ Homework
\end{center}

Compute the following anti-derivatives.  See if you identify any patterns.
\begin{multicols}{3}
\bfseries{\begin{Large}
	\begin{enumerate}
		\item $\displaystyle\int \cfrac{1}{1+x^2}\; dx$
		\vspace{3cm}
		\item $\displaystyle\int \cfrac{1}{1+4x^2}\; dy$
		\vspace{3cm}
		\item $\displaystyle\int \cfrac{1}{1+3x^2}\; dx$
		\vspace{3cm}
		\item $\displaystyle\int \cfrac{1}{1+2x^2}\; dx$
		\vspace{3cm}
		\item $\displaystyle\int \cfrac{1}{1+x^2}\; dx$
		\vspace{3cm}
		\item $\displaystyle\int \cfrac{1}{4+x^2}\; dx$
		\vspace{3cm}
		\item $\displaystyle\int \cfrac{1}{3+x^2}\; dx$
		\vspace{3cm}
		\item $\displaystyle\int \cfrac{1}{2+x^2}\; dx$
		\vspace{3cm}
		\item $\displaystyle\int \cfrac{1}{9+4x^2}\; dx$
		\vspace{3cm}
		\item $\displaystyle\int \cfrac{1}{25+9x^2}\; dx$
		\vspace{3cm}
		\item $\displaystyle\int \cfrac{1}{4x^2+16}\; dx$
		\vspace{3cm}
		\item $\displaystyle\int \cfrac{1}{1+x^2}\; dx$
	\end{enumerate}
\end{Large}}

\end{multicols}


	
\end{document}
