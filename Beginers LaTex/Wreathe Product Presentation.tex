\documentclass[12pt, letterpaper]{report}
\usepackage[margin=0.75in]{geometry}
\usepackage{url}
\usepackage[utf8]{inputenc}
\usepackage{blindtext}
\usepackage[usenames,dvipsnames]{xcolor} %adds mucho colors
\usepackage{amsmath}
\usepackage{amsthm}
\usepackage{graphicx}
\usepackage{enumitem} %allows to resume enumeration
\usepackage{tabu} %allows tables using tabu
\usepackage{standalone}

\usepackage{tikz}
\usetikzlibrary{decorations.pathreplacing}
\usepackage{tikz-network}
\usetikzlibrary{matrix}
\usepackage{pgfplots}

\usetikzlibrary{arrows.meta,shapes,automata,backgrounds,petri,positioning}
\usetikzlibrary{decorations.pathmorphing}
\usetikzlibrary{decorations.shapes}
\usetikzlibrary{decorations.text}
\usetikzlibrary{decorations.fractals}
\usetikzlibrary{decorations.footprints}
\usetikzlibrary{calc}
\usetikzlibrary{spy}

\usepackage{parskip} %something about paragraph lines
%defines columns
\usepackage{multicol}
\setlength{\columnsep}{1cm}
\usepackage{wrapfig} %able to insert pictures in text
\usepackage{gensymb} %allows the degree symbol for weather or angles
\usepackage{amssymb}
\usepackage{epigraph} %adds little quote start of chapter

\usepackage{systeme}%something about allowing system of equations stuff

\usepackage{pdfpages} %allows for import of separate files

\usepackage[framemethod=TikZ]{mdframed} %allows framed definitions 

\def\acts{\curvearrowright}
%%%%%%%%%%THEOREM BOXES%%%%%%%%%%%%%%%%%%
\newcounter{theo}\setcounter{theo}{0}
%\renewcommand{\thetheo}{\arabic{chapter}.\arabic{theo}}
\newenvironment{theo}[2][]{%
						\refstepcounter{theo}%
						\ifstrempty{#1}%
						{\mdfsetup{%
						frametitle={%
						\tikz[baseline=(current bounding box.east),outer sep=0pt]
						\node[anchor=east,rectangle,fill=JungleGreen!55]
						{\strut Theorem~\thetheo};}}
						}%
					{\mdfsetup{%
					frametitle={%
					\tikz[baseline=(current bounding box.east),outer sep=0pt]
					\node[anchor=east,rectangle,fill=JungleGreen!55]
					{\strut Theorem~\thetheo:~#1};}}%
						}%
					\mdfsetup{innertopmargin=10pt,linecolor=JungleGreen!55,%
					linewidth=2pt,topline=true,%
					frametitleaboveskip=\dimexpr-\ht\strutbox\relax
					}
\begin{mdframed}[]\relax%
\label{#2}}{\end{mdframed}}
%%%%%%%%%%%%%%%%%%%%%%%%%%%%%%%%%%

%%%%%%%%%%DEFINTION BOXES%%%%%%%%%%%%%%%%%%
\newcounter{defi}[chapter] \setcounter{defi}{0}
\renewcommand{\thedefi}{\arabic{chapter}.\arabic{defi}}
\newenvironment{defi}[2][]{%
						\refstepcounter{defi}%
						\ifstrempty{#1}%
						{\mdfsetup{%
						frametitle={%
						\tikz[baseline=(current bounding box.east),outer sep=0pt]
						\node[anchor=east,rectangle,fill=WildStrawberry!55]
						{\strut Definition~\thedefi};}}
						}%
					{\mdfsetup{%
					frametitle={%
					\tikz[baseline=(current bounding box.east),outer sep=0pt]
					\node[anchor=east,rectangle,fill=WildStrawberry!55]
					{\strut Definition~\thedefi:~#1};}}%
						}%
					\mdfsetup{innertopmargin=10pt,linecolor=WildStrawberry!55,%
					linewidth=2pt,topline=true,%
					frametitleaboveskip=\dimexpr-\ht\strutbox\relax
					}
\begin{mdframed}[]\relax%
\label{#2}}{\end{mdframed}}
%%%%%%%%%%%%%%%%%%%%%%%%%%%%%%%%%%

%%%%%%%%%%%%%%%%Steps%%%%%%%%%%%%%%%

\newcounter{proc}[section] \setcounter{proc}{0}
\renewcommand{\theproc}{ }
\newenvironment{proc}[2][]{%
						\refstepcounter{proc}%
						\ifstrempty{#1}%
						{\mdfsetup{%
						frametitle={%
						\tikz[baseline=(current bounding box.east),outer sep=0pt]
						\node[anchor=east,rectangle, rounded corners,fill=RoyalBlue!60]
						{\strut PS~};}}
						}%
					{\mdfsetup{%
					frametitle={%
					\tikz[baseline=(current bounding box.east),outer sep=0pt]
					\node[anchor=east,rectangle, rounded corners,fill=RoyalBlue!60]
					{\strut PS~:~#1};}}%
						}%
					\mdfsetup{roundcorner=5pt, innertopmargin=10pt,linecolor=RoyalBlue!60, %
					linewidth=2pt,topline=true,%
					frametitleaboveskip=\dimexpr-\ht\strutbox\relax
					}
\begin{mdframed}[]\relax%
\label{#2}}{\end{mdframed}}
%%%%%%%%%%%%%%%%%%%%%%%%%%%%%%%%%%

%%%%%%%%%%%%%%%%%%%%%%%%%%%%%%%%%%

\def\firstcircle{(0,0) circle (1.5cm)}
\def\secondcircle{(0:2cm) circle (1.5cm)}

\colorlet{circle edge}{blue!50}
\colorlet{circle area}{blue!20}

\tikzset{filled/.style={fill=circle area, draw=circle edge, thick},
    outline/.style={draw=circle edge, thick}}
%%%%%%%%%%%%%%%%%%%%%%%%%%%%%%%%%%

\usetikzlibrary{arrows}
\usepackage{mathtools,etoolbox}
\DeclarePairedDelimiterX{\abs}[1]{\lvert}{\rvert}{\ifblank{#1}{{}\cdot{}}{#1}}

%defined theorems and definitions section
\theoremstyle{definition}
\newtheorem{defn}{Definition}

\theoremstyle{definition}
\newtheorem{ndefn}{Naive Definition}

\theoremstyle{definition}
\newtheorem{ex}{Example}

\theoremstyle{definition}
\newtheorem{DIY}{DIY}

\theoremstyle{definition}
\newtheorem{post}{Postulate}

\theoremstyle{theorem}
\newtheorem{thm}{Theorema}

\theoremstyle{theorem}
\newtheorem{cor}{Lemma}

\theoremstyle{remark}
\newtheorem{rmk}{Remark}

\begin{document}

\begin{center}
\fbox{\fbox{\parbox{5.5in}{\centering
The Wreath Product}}}
\end{center}

\noindent \textbf{Motivation:}
\begin{itemize}
	\item Abstract Algebra as a class itself is usually represented, to the naive spectator, as an archetypical embodiment of, a Rubik's cube. 
		\begin{itemize}
			\item Symmetries of a Rubik's Cube
			\item Rubik's Cube Group
		\end{itemize}
	
	\item Rooted Trees
		\begin{itemize}
			\item Sylow p-subgroup of $S_{p^n}$ for $n\geq 1$ (Generalized Symmetric Group)
			\item Young's Lattice (Bratteli's Diagrams)
		\end{itemize}

\end{itemize}

{ \rule{\textwidth}{3pt}
}
\noindent Consider the following problem:
\begin{center}
	\emph{Classify the groups of order $18$.}
\end{center}

$\to$ By the Sylow Theorems we have that $|G|=2\cdot 3^2$. 

So it follows by the Third Sylow Theorem that $n_2 \in {1,3,9}$ and $n_3\in {1,2}$

Now, it must follow that $n_3=1$ as $2 \not\equiv 1 (\mbox{mod} 3)$. Thus, we must have a subgroup of $G$ such that the subgroup is of order 9.

Let $H$=(subgroup of order $9$); therefore up to isomorphism we have $$H \cong \mathbb{Z/}3\mathbb{Z} \times \mathbb{Z/}3\mathbb{Z} \hspace{1cm} \mbox{ or } \hspace{1cm} H\cong \mathbb{Z/}9\mathbb{Z}$$.
We could proceed and find the groups of order 18 up to isomorphism by considering cases, but this was done in [2]. 

So, we instead turn to the (internal) semi-direct product we have $$G \cong H \rtimes \mathbb{Z/}2\mathbb{Z}$$.
Now we have the following classifications based on whether $\mathbb{Z/}2\mathbb{Z}$ acts on $H$ via ${+1}$ or ${-1}$.

$$H \cong \mathbb{Z/}3\mathbb{Z} \times \mathbb{Z/}3\mathbb{Z} \rtimes \mathbb{Z/}2\mathbb{Z}$$ 
\textbf{Under $+1$:}

$G\cong \bigg(\mathbb{Z/}3\mathbb{Z} \times \mathbb{Z/}3\mathbb{Z}\bigg)\times\mathbb{Z/}2\mathbb{Z} \cong  \mathbb{Z/}3\mathbb{Z} \times \mathbb{Z/}6\mathbb{Z}$ \hspace{1cm} or \hspace{1cm} $G \cong \mathbb{Z/}3\mathbb{Z} \times S_3$ (\emph Wreath Product)

\textbf{Under $-1$:}

$G\cong \bigg(\mathbb{Z/}3\mathbb{Z} \times \mathbb{Z/}3\mathbb{Z}\bigg)\rtimes\mathbb{Z/}2\mathbb{Z}$ (\emph Frobenius Group)

$$H \cong \mathbb{Z/}9\mathbb{Z} \rtimes \mathbb{Z/}2\mathbb{Z}$$ 
\textbf{Under $+1$:}

$G\cong \bigg(\mathbb{Z/}9\mathbb{Z} \times \mathbb{Z/}2\mathbb{Z} \bigg) \cong  \mathbb{Z/}18\mathbb{Z}$ \hspace{1cm} or \hspace{1cm} $G \cong D_{18}$

\newpage

Now we look closer at $\mathbb{Z/}3\mathbb{Z} \times S_3$ since it is our first encounter with a Wreath Product, but first a definition:

\begin{defi}[Wreath Product]{}
	Consider $G$ and $H$ as groups, then $$\underbrace{G \times G \times \ldots \times G}_{\text{H - times}} \rtimes H $$
	Where the (internal) semi-direct product has $H\acts G^H$ and $H\to Aut(G^H)$.
\end{defi}

\begin{ex}
	Sylow Subgroups of Symmetric Groups:
	\begin{center}
		\emph{Consider finding the Sylow 3-subgroups of $S_{27}$.}
	\end{center}
\end{ex}

\vspace{5cm}

\begin{ex}
	Sylow 2-subgroups of $S_{13}$.
\end{ex}

\vspace{5cm}

Proposition: Every $p-$subgroup is within some $S_n$.  In particular, it must be in one of the wreaths.

This proposition might help us try and classify all p-subgroups, but this wouldn't be too efficient, so we instead focus on another aspect of the Wreath Product for now.  We will now define the Wreath Product from a more generalized perspective.

\newpage

\begin{defi}[Wreath Product (D\&F)]{}
	Let $G$ and $L$ be groups.  Let $n \in \mathbb{Z}^+$, and $\rho: G \to S_n$ be a homomorphism. 
	Define $$H := \underbrace{L \times L \times L \ldots \times L}_{n-times}$$
	Recall that $\psi: S_n \to Aut(H)$ is an injective homomorphism by $$\psi_\pi(l_1,l_2, \ldots, l_n) = (l_{\pi^{-1}(1)},l_{\pi^{-1}(2)}, \ldots, l_{\pi^{-1}(n)}) \mbox{ and } \psi_{\pi_{1}} \circ \psi_{\pi_{2}} = 
	\psi_{\pi_{1} \pi_{2}}$$
	where $\pi \in S_n$ is a fixed.
	
	Then $\psi \circ \rho$ is a homomorphism from $G \to Aut(H)$.  We define the wreath product of $L$ by $G$ as the internal semi-direct product $H \rtimes G$ with respect to this homomorphism.  We denote the wreath by: 
	$$L \wr G$$
	
	If $\rho$ is the left regular representation of $G$, then $$\rho_{g}: h \to gh \hspace{1cm}\forall h \in G$$ 
	and $$(\rho_{g}f)(x)=f(\rho_{g^-1}(x))=f(g^{-1}(x))$$
	
\end{defi}

Now considering to arbitrary elements under this new group, we define the operation $(\star)$ for this group as : $$(f,b) \star (f',b')=(fbf',bb') $$
where we have $(fbf')(\gamma)=f(\gamma)f'(b^{-1}\gamma)$.

Now, as coordinates, we have the following:

$$(f,b) \star (f',b')= \bigg((f(1),f(2),f(3),\ldots f(n)) , b \bigg) \star \bigg((f'(1),f'(2),f'(3),\ldots f'(n)) , b \bigg)$$

\vspace{2cm}

\newpage

Equipped with this we can now understand the symmetries of a Rubik's cube with a bit more depth, we can find a representation of the Generalized Symmetric Group, and we can show an analogous theorem to Fermat's Little theorem with Sylow $p-$subgroups.  For this we consider groups $A,B$ such that they are of finite order.

\begin{defi}[Wreath Product as Exponentiation of Sets]{}
	The wreath product of $A$ by $B$ is defined by the semi-direct product group $A^{B} \rtimes B$, where $B$ acts on the index set of $A^B$ via left multiplication.  We write any element of $A\wr B$ in the canonical form where for $f \in A^B$ such that $f(b)=a$ if $b=b_0\in B$ which we can write as $\sigma_a(b_0)$ and more explicitly as $$\sigma_{a_{1}}(b_1)\ldots \sigma_{a_{n}}(b_n)\tau(b) $$
\end{defi}


\begin{theo}{}
Consider two finite groups, A and B.
Let $|B|=p$, and $gcd(|A|,p)=1$ Then, the number of p-Sylow subgroups in the wreath product $A\wr B$ is $|A|^{p-1}$. Thus we have $|A|^{p-1} \equiv 1 (\mbox{mod } p)$.
\end{theo}
\begin{theo}{}
Let $A,B$ be finite groups, then the normalizer of $B$ in $A \wr B$ is equal to $C \times B$, where $C$ is the subgroup defined as: $$C:=\{\sigma_a(b_1)\ldots \sigma_a(b_n): a \in A\}$$ and $B=\{b_1,\ldots,b_n\}.$

\end{theo}

\begin{theo}{}
	Let $A,B$ be finite solvable groups of order $a,b$ with $gcd(a,b)=1$.  Then, the subgroups of order $b$ in $A\wr B$ are conjugate and, are $a^{b-1}$ in number.
\end{theo}

\newpage

\begin{center}
\textbf{References}

\begin{verbatim}

[1] Group Theory 19: Wreath Products - YouTube. 
https://www.youtube.com/watch?v=2Ik9m84AT2E. 

[2] Groups of Order 18. https://Www.math.utk.edu/~Finotti/f06/m455/g18.Pdf. 

[3] mlbaker. Group Theory: Semidirect Products, Extensions. 
https://www.youtube.com/watch?v=Xz7_QJ-sIVk&amp;t=1126s. 

[4] Rooted Trees and Iterated Wreath Products of Cyclic Groups. 
https://core.ac.uk/download/pdf/81946483.pdf. 

[5] Wells, Charles. "Automorphisms of Group Extensions.” Transactions of 
The American Mathematical Society, 23 Apr. 2014, 
https://www.academia.edu/4744084/Automorphisms_of_Group_Extensions. 

[6] Wreath Products, Sylow's Theorem and Fermat's Little Theorem. 
https://www.isibang.ac.in/~sury/wreathpf.pdf. 


\end{verbatim}

\end{center}




	
\end{document}
